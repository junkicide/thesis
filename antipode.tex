\documentclass[12pt,a4paper,draft]{article}
\usepackage{amsthm,amsmath,amssymb,bbm}
\usepackage{hyperref}
\usepackage{cleveref}
\usepackage{mathtools}
\usepackage{stmaryrd}
\usepackage{xcolor}
\usepackage[all]{xy}
\usepackage{bbm}
\usepackage{amsfonts,amscd,amssymb,amsmath,amsthm,hyperref,mathrsfs,multirow,xcolor,lscape,ifthen,mathtools,graphicx,fancyhdr,enumerate}
\usepackage[thinlines]{easytable}
\setcounter{tocdepth}{1}
\usepackage{tikz}
\usepackage{tikz-qtree}
\usetikzlibrary{matrix}
\usetikzlibrary{calc}
\usepackage{listings}
\lstset{language=GAP,rangeprefix=\#\&\ ,rangesuffix=\ \#\&,includerangemarker=false,   inputencoding=latin1,
   basicstyle=\footnotesize\ttfamily,}
\newcommand{\red }[1]{{\color{red}#1}}
\newcommand{\blue}[1]{{\color{blue}#1}}
\newcommand{\green}[1]{{\color{green}#1}}
\newcommand{\brown}[1]{{\color{brown}#1}}
\newcommand{\purple}[1]{{\color{purple}#1}}

 \newtheorem{theorem}{Theorem}[section]
 \newtheorem{Lem}[theorem]{Lemma}
 \newtheorem{Prop}[theorem]{Proposition}
 \newtheorem{Cor}[theorem]{Corollary}
 \newtheorem{Conj}[theorem]{Conjecture}
\theoremstyle{remark}
 \newtheorem{Expl}[theorem]{Example}
\newtheorem{Expls}[theorem]{Examples}
 \newtheorem{Rem}[theorem]{Remark}
\newtheorem{Qu}[theorem]{Question}
\theoremstyle{definition}
 \newtheorem{Def}[theorem]{Definition}
\numberwithin{equation}{section}
\newcommand \N{\mathbb N}
\newcommand\ol[1]{\overline{#1}}
\newcommand\ex{\text{exp}}
\newcommand\braid{\sigma}
\newcommand\tr{\operatorname{tr}}
\newcommand\End{\operatorname{End}}

\newcommand \hit {\triangleright}

\newcommand\bhit{\blacktriangleright}
\newcommand\inv{^{-1}}
\def \HM#1.#2.#3.#4.{{^{#1}_{#3}\mathcal M^{#2}_{#4}}}
\newcommand\lYD[1]{{^{#1}_{#1}\mathcal{YD}}}

\newcommand\id{\operatorname{id}}

\newcommand\ot{\otimes}
\newcommand {\ou}[1]{\underset{{#1}}{\otimes}}
\newcommand {\co}[1]{\mathrel{\mathop{\Box}_{#1}}}

\newcommand\Res{\operatorname{Res}}
\newcommand\Ind{\operatorname{Ind}}
\newcommand\Coind{\operatorname{Coind}}
\newcommand\Stab{\operatorname{Stab}}
\newcommand\Rep{\operatorname{Rep}}
\newcommand\Irr{\operatorname{Irr}}
\newcommand\Vect{\texttt{Vec}}
\newcommand\Tr{\operatorname{Tr}}
\newcommand\lquot{\backslash}
\newcommand\Img{\operatorname{Im}}
\newcommand\ev{\operatorname{ev}}
\newcommand\coev{\operatorname{coev}}
\newcommand\Ext{\text{Ext}}
\newcommand\one{\mathbbm{1}}
\newcommand\Hom{\operatorname{Hom}}
\newcommand\Mod{\operatorname{Mod}}

\newcommand\Indtobim{\mathcal F}
\newcommand\IndtoYD{\mathcal G}

\newcommand \Q{\mathbb Q}
\newcommand\CC{\mathbb C}
\newcommand\RR{\mathbb R}
\newcommand\CCu{\CC^\times}
\newcommand\ZZ{\mathbb Z}
\newcommand\M {\mathcal M}
\newcommand \A{\mathcal A}
\newcommand \B{\mathcal B}
\newcommand \D{\mathcal D}
\newcommand \C{\mathcal C}
\newcommand\CTR{\mathcal Z}
\newcommand\galaut{\sigma}
\newcommand\adams{\psi}
\newcommand\auta{\psi}
\newcommand\autb{\widetilde\psi}
\newcommand\doubleadams{\widehat\psi}

\newcommand\wreath{\wr}

\newcommand\semidir\rtimes
\newcommand{\tmt}[4]{\left({#1\atop #3}{#2\atop #4}\right)}
\newcommand{\cmt}[2]{\left({#1\atop #2}\right)}
\newcommand\Inf{\operatorname{Inf}}
\newcommand\Gal{\operatorname{Gal}}
\newcommand\ab{{\operatorname{ab}}}
\newcommand {\gb}{\overline{g}}
\newcommand {\hb}{\overline{h}}
\newcommand {\kb}{\overline{k}}
\title{tqft and square of the antipode}


\begin{document}
Noah Snyder suggested the idea that it might be possible to prove that $S^2=\id$ for semisimple, cosemisimple Hopf algebras, using an approach involving framed TQFTs with corners and defects. The idea is to use the fiber functor to vec (or equivalently a rank 1 module category), and "pull theories across defects".

\section{Tensor categories}
There are various notions for duality in the 3-category TC of tensor categories. 
\begin{enumerate}
	\item The dual of a tensor category $\C$ as an object of the symmetric monoidal 3-category TC
	\item Duals of objects within $\C$
	\item The adjoint of the $\C$-$\D$-bimodule category seen as a $1$-morphism of TC
	\item adjoints of functors on bimodule categories
\end{enumerate}
\begin{Def}
An adjunction between functors $F$ and $G$, $F  \dashv G: \A \rightarrow \B$ is a couple of natural transformations, the unit $\eta: id_{\B} \rightarrow G\circ F$ and the counit $\epsilon: F \circ G \rightarrow id_{\A} $, such that for an object $ X\in \A$ and $Y \in \B$ the following diagrams commute, 
\begin{equation} 
\xymatrix{
GX \ar[r]^{G\eta_{X} }\ar[rd]_{\id_{GX}} & GFGX \ar[d]^{\epsilon_{GX}} \\
 & GX
}	
\xymatrix{
FY \ar[r]^{\eta_{FY}}\ar[rd]_{\id_{FY}} & FGFY \ar[d]^{F\epsilon_{Y}} \\
 & FY
}
\end{equation}
This may be expressed as the following equations, \begin{equation}\label{adj1}
	\id_G = (\epsilon \odot id_G)\circ(id_G \odot \eta),
\end{equation}
\begin{equation}\label{adj2}
	\id_F = (\id_F\odot \epsilon) \circ (\eta \odot \id_F).
\end{equation}
\end{Def}

We have the canonical natural isomorphism \begin{equation}
\Hom_\A (FY, X) \cong \Hom_\B (X, GY),
\end{equation}
from which we interpret $F$ as the left adjoint and $G$ as the right adjoint.
Considering functors as 1-morphisms and natural transformations as 2-morphisms in a 2-category, we may make a similar definition, where $\odot$ will denote horizontal composition of 2-morphisms.
If we replace $\odot$ by the tensor product $\otimes$ in equations (\ref{adj1}) and (\ref{adj2}), then the above definition is the definition of rigidity for a monoidal category $(\C, \otimes, \one)$ if we consider $F$ and $G$ as objects in the category $\C$. 

Let $\C$ be a monoidal category, considered as a 2-category with 1 object, *. Then there is a 2-functor $F:\C \rightarrow$ Cat, the 2-category of categories, which is given by
\begin{align}
	F(*) &= \C\\
	F(A:*\rightarrow *) &= - \otimes A \in \End(\C)\\
	F(f: A\Rightarrow B) &= -\otimes f :-\otimes A \rightarrow -\otimes B  
\end{align}
Note that  
\begin{equation}
 	 F(*\xrightarrow{B\circ A}*) = \C\xrightarrow{-\otimes A\otimes B}\C,
 \end{equation} 
 but by convention of [dsps] $A\otimes B := B\circ A$ for 1-morphisms $*\rightarrow *$.
(A should probably be an algebra object?)

Let Alg be the 2-category with objects algebras, 1-morphisms bimodules and 2-morphisms bimodule maps. Then there is a 2-functor sending an algebra A to mod-A in Cat.

(How is any linear category a Vec bimodule category?)

to be continued...

\section{Misc}

An object in a symmetric monoidal $(\infty, 3)$- category is called 1-dualizable if it is dual in the usual monoidal categorical sense. It is 2-dualizable if there are series of adjunctions 
\begin{equation}
 	\dots\ev^{LL} \dashv \ev^L \dashv \ev \dashv \ev^R \dashv \ev^{RR}\dots 
 \end{equation} 
 and 
 \begin{equation}
 		\dots\coev^{LL} \dashv \coev^L \dashv \coev \dashv \coev^R \dashv \coev^{RR}\dots. 
 \end{equation}
An object is 3-dualizable if for every adjunction $(F, G, u, v)$ in the above chains, the unit $u$ and the counit $v$ are part of an infinite chain of adjunctions.

A symmetric monoidal 3-category $\C$ is 1-dualizable if every object has a dual in the symmetric monoidal 1-category sense. It is 2-dualizable if it is 1-dualizable and if every 1 morphism has a left and right adjoint.\\
It is 3-dualizable if it is 1- and 2-dualizable and every 2-morphism has a left and right adjoint.

\begin{Def}
\textbf{Cobordism hypothesis:} $n$-dimensional local framed topological field theories with target a symmetric monoidal $(\infty, n)$-category $\C$ are in one to one correspondence with the n-dualizable objects of $\C$; in fact, the space of such field theories homotopy equivalent to the space of n-dualizable objects of $\C$.
\end{Def}
An n-dualizable (object in an)$(\infty, n)$-category is also called a fully dualizable (object in an) $(\infty, n)$-category.

\subsection{$n$-framings}

An $n$-framed $k$-manifold is a $k$-manifold $M$ with a trivialization $\tau$ of $TM +\RR^{n-k}$, the $(n-k)$-fold stabilization of the tangent bundle of $M$. One can realize this by immersing M into $\RR^n$, and taking a trivialization of the tangent bundle along with a trivialization of the normal bundle of the immersion.






\end{document}